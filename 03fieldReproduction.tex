\documentclass[11pt,a4j]{jreport}

\usepackage{comment}
\usepackage{float}
\usepackage{color}
\usepackage{multicol}
\usepackage{multirow}
\usepackage[dvipdfmx]{pict2e}
\usepackage{wrapfig}
\usepackage{graphicx}
\usepackage{bm}
\usepackage{url}
\usepackage{underscore}
\usepackage{colortbl}
\usepackage{tabularx}
\usepackage{fancyhdr}
\usepackage{ulem}
\usepackage{cite}
\usepackage{amsmath,amssymb,amsfonts}
\usepackage{algorithmic}
\usepackage{textcomp}
\usepackage{xcolor}
\usepackage[ipaex]{pxchfon}
\usepackage{pdfpages}
\usepackage{subcaption}
\usepackage{array}
\usepackage{adjustbox}
\usepackage{lipsum}

\usepackage[number-unit-product=~]{siunitx}

\usepackage[top=30truemm,bottom=30truemm,left=25truemm,right=25truemm]{geometry}

\renewcommand{\arraystretch}{1.2}

\begin{document}

\chapter{実験音場の生成}

第2章にて、コンサートホールのステージ上における反射音の到来方向を定量化する指標として$\mathrm{ST_{Early,dir}}$および$\mathrm{ST_{Late,dir}}$を定義し、その実測値をもとに演奏実験に用いるために生成する音場の目標値を設定した。

本章では、音場生成システムAFCを用いた実験音場の生成に関して述べる。まず、本研究で音場生成に用いるシステムである音場支援システムAFCとそれを導入した実験室(半無響室)について紹介する。続いて、生成する音場での方向別STおよび周波数特性の調整を含む残響時間とSTの調整方法について、音場の生成における考え方とそれに基づく具体的な調整値について説明する。最後に、コンサートホールを模して生成した音場を基準音場として、基準音場の音響特性と、基準音場から方向特性を変化させた音場の音響特性を測定した結果結果を示す。


\section{音場支援システムAFC}

\subsection*{音場支援システムの概要}
第1章4節でも触れたように、音場支援システム(Active Field Control Enhance、AFC)は、一つの空間において多様な演目を最適な響きの中で行いたいというモチベーションから開発されたシステムであり、最新の電気音響・信号処理技術を用いて、室内の響きや空間の拡がり・音量感などの聴感印象を自然に変化させることができる\cite{AFC}。音源自体に人工的なリバーブを付加して異なる音の印象を作り出す手法とは異なり、楽器や歌声の自然な聴こえ方を保ちながら、その空間に拡がる音の残響感や音量感をコントロールし、用途に適した音響空間を提供できる点にその特徴がある。

マイクロフォンとスピーカーを配置して音場を生成するシステムにおける信号処理の方式は、マイクで収音した音をスピーカーから再生し、その直接音に空間固有の響きが加わった音をもう一度収音する音響的フィードバックを利用する室内音場制御方式と、収音した音に様々な実測インパルス応答のデータを畳み込むことで任意の音場を再現する音場合成方式の二つの方式に大別され、AFCではこれら二つの方式を兼ね備えたハイブリッドなシステムとして構成されている。

%=======================================================================
\newpage
\subsection*{音場支援システムの仕組み}
響きの聴覚印象には初期反射音と残響音(後期反射音)がそれぞれ異なる影響を与えていると考えられ、AFCにおける音響信号の処理系統も、初期反射音制御部と後期反射音制御部に大きく分かれてそれぞれの反射音成分をある程度独立に制御している。この概略図を図\ref{fig:AFCにおける初期反射音と後期反射音の独立制御の様子}に示す。

どちらの制御部においても、マイクロホンで収音した音に対してインパルス応答の畳み込みによる響きの付加を行ってからスピーカーでその信号を再生し、響きを増幅させる点は共通している。初期反射音制御部では演奏者の付近に設置した指向性マイクによって主に直接音と舞台上からの初期反射音を収音して初期反射音成分に相当する響きをスピーカーから再生させるのに対し、後期反射音制御部では演奏者から遠方に付置した全指向性マイクによって主に空間の残響音とスピーカーから再生される音響フィードバック成分を収音して後期反射音に相当する響きをスピーカーから再生させている。

初期・後期どちらの制御部でも室内音場制御方式と音場合成方式を組み合わせた制御方式となっており、AFC内部での信号処理の流れは次の通りである。まず、室内に配置された各マイクに収音された信号がbusと呼ばれる信号処理系に割り振られる。続いて、busに割り振られた音響信号に対して「FIRフィルタの重畳」「遅れ時間の付与」「音量の増減」「イコライザによる周波数特性の変化」が組み合わせて適用される。そして、busによって処理された音響信号は室内に配置されたスピーカーに割り振られて出力される。このbusごとの信号処理の概要を図\ref{fig:busごとの信号処理の概要}に示す。

AFCではスピーカーから再生された音がマイクへ戻る音響的フィードバックを利用するため、生成する響きを大きくしようとするとループゲインが増大してハウリングが発生しやすくなる恐れがあることが重要な技術的課題となっている。そこでAFCでは、マイクからbusへのルーティングを時変的に切り替えることによってハウリングの発生を抑制する処理であるヤマハ独自の特許技術EMR(Electronic Microphone Rotator)\cite{清水1996EMR}が使用されている。AFCにおける信号処理系は4つのbusを1組の「System」として管理され、EMRは主に後期反射音の制御用に割り当てられたSystemにおいて適用される。このSystemごとの信号処理の概要を図\ref{fig:Systemによる信号処理の概要}に示す。

\vspace{\stretch{1}}

\begin{figure}[H]
  \centering
  \includegraphics[width=0.8\linewidth]{images/AFCsystemOverView.png}
  \caption{AFCにおける信号処理系統の概要}
  \label{fig:AFCにおける信号処理の概要}
\end{figure}

\vspace{\stretch{1}}
%=======================================================================
\newpage

\vspace*{\stretch{1}}
\begin{figure}[H]
  \centering
  \includegraphics[width=0.8\linewidth]{images/AFCbusProcessing.png}
  \caption{busごとの信号処理の概要}
  \label{fig:busごとの信号処理の概要}
\end{figure}

\vspace*{\stretch{1}}

\begin{figure}[H]
  \centering
  \includegraphics[width=0.8\linewidth]{images/AFCsystemProcessing.png}
  \caption{Systemによる信号処理の概要}
  \label{fig:Systemによる信号処理の概要}
\end{figure}

\vspace*{\stretch{1}}
%=======================================================================

\newpage
AFCを用いることによりカルテットによる演奏実験および現実のホールでの測定と同様の音場測定が実施可能なある程度広い範囲に現実に即した自然な音場を生成することが可能であり、また導入する室をもともとの響きのない半無響室とすることで大きな制御幅を得ることが期待できる。

音場の支援を目的として開発されたシステムである。AFCは、音場の再現において重要な要素である初期反射音と後期反射音をそれぞれ制御するための2つの部分から構成されている。初期反射音制御部は、音場の初期反射音を制御するためのFIRフィルタを用いて音場の初期反射音の到来方向を制御する。後期反射音制御部は、音場の後期反射音を制御するためのFIRフィルタを用いて音場の後期反射音の到来方向を制御する。AFCの概要を図\ref{fig:AFCの概要}に示す。

\subsection*{実験室における音場の可変幅の確認}
実験室に響きを付加するにあたり、目標とした音場が実験室で実現可能な響きの幅、すなわち最小の響きの条件と最大の響きの範囲に収まることを確認する必要がある。

AFCは響きを増幅するシステムであるため、実験室で実現可能な最小の響きは、AFCをオフにしたときの響きとなる。AFCをオフししたときの$\mathrm{ST_{Early,dir}}$、$\mathrm{ST_{Late,dir}}$を表\ref{table:響き小の方向別ST}に示す。

\begin{table}[H]
  \centering
  \caption{AFCをオフにしたときの$\mathrm{ST_{Early,dir}}$、$\mathrm{ST_{Late,dir}}$}
  \label{table:響き小の方向別ST}
  \begin{tabular}{c|cccccc} \hline \hline
    \begin{tabular}{c}
    
    \end{tabular}&
    \begin{tabular}{c}
      Front
    \end{tabular}&
    \begin{tabular}{c}
      Back
    \end{tabular}&
    \begin{tabular}{c}
      Left
    \end{tabular}&
    \begin{tabular}{c}
      Right
    \end{tabular}&
    \begin{tabular}{c}
      Up
    \end{tabular}&
    \begin{tabular}{c}
      Down
    \end{tabular} \\ \hline
    $\mathrm{ST_{Early,dir}}$ (dB) & -24.2 & -25.5 & -24.2 & -25.9 &-25.9 & -23.3 \\
    $\mathrm{ST_{Late,dir}}$ (dB) & -65.5 & -66.7 & -67.0 &-65.6 & -66.6 & -64.2 \\ \hline \hline
  \end{tabular}
\end{table}

響きを増やすことに関してはある程度自由度を持って調整することができるが、AFCでつける響きを大きくしていく、つまり再生音の音量を上げていくと、ある音量を超えたときに不自然な音色の変化(カラレーション)が発生し始める。カラレーションが起こり始める音量の条件はイコライザの設定等様々な要素によって変化するため最大の響きを厳密に調べることは困難である。
    
実験室での可変幅の大まかな検討のため、イコライザによる周波数特性の手動調整を行わずに仮の設定でAFCをオンにし、カラレーションの生じ始めを聴感的に確認し、その直前の音量の設定における$\mathrm{ST_{Early,dir}}$および$\mathrm{ST_{Late,dir}}$を測定した。その結果を表\ref{table:響き大の方向別ST}に示す。

\begin{table}[H]
  \centering
  \caption{AFC仮調整時の$\mathrm{ST_{Early,dir}}$、$\mathrm{ST_{Late,dir}}$}
  \label{table:響き大の方向別ST}
  \begin{tabular}{c|cccccc} \hline \hline
    \begin{tabular}{c}
    
    \end{tabular}&
    \begin{tabular}{c}
      Front
    \end{tabular}&
    \begin{tabular}{c}
      Back
    \end{tabular}&
    \begin{tabular}{c}
      Left
    \end{tabular}&
    \begin{tabular}{c}
      Right
    \end{tabular}&
    \begin{tabular}{c}
      Up
    \end{tabular}&
    \begin{tabular}{c}
      Down
    \end{tabular} \\ \hline
    $\mathrm{ST_{Early,dir}}$ (dB) & -16.4 & -18.0 & -16.7 & -17.3 & -19.9 & -16.1\\
    $\mathrm{ST_{Late,dir}}$ (dB) & -15.8 & -15.9 & -16.0 & -15.8 & -19.2 & -15.0\\ \hline \hline
  \end{tabular}
\end{table}

    

これらの結果とコンサートホールでの実測値から、実験室における$\mathrm{ST_{Early,dir}}$と$\mathrm{ST_{Late,dir}}$はコンサートホールでの実測値と同程度の値を十分取ることができると考えられ、詳細な設定により所望の音場を実現しうることが期待できる。
%=====================================================================================

  \section{生成する基準音場の目標値}
    第3章における実際のコンサートホールでの測定結果について、標準的な演奏位置である舞台中央・前方側の4点(S01・S02・S05・S06)における
    全ホールの$\mathrm{ST_{Early,dir}}$および$\mathrm{ST_{Late,dir}}$は図\ref{fig:舞台中央・前方側4点の方向別ST}
    に示す通りである。実線が測定値の平均、色付きの領域が測定値の標準偏差を平均値に加算またはまたは減算した区間(1シグマ区間)を表す。

    \begin{figure}[H]
      \begin{minipage}
        [b]{.5\linewidth}
        \centering
        \includegraphics[width=1\linewidth]{images/targetFieldEarlywithLegend.png}
        \subcaption*{$\mathrm{ST_{Early,dir}}$}
      \end{minipage}%
      \begin{minipage}
        [b]{.5\linewidth}
        \centering
        \includegraphics[width=1\linewidth]{images/targetFieldLatewithLegend.png}
        \subcaption*{$\mathrm{ST_{Late,dir}}$}
      \end{minipage}
    \centering
    \label{fig:舞台中央・前方側4点の方向別ST}
    \caption{舞台中央・前方側4点の方向別ST}
    \end{figure}

    この結果は第3章において検討した実験室における音場の可変幅の中で生成しうると期待できるため、基準音場として生成する音場は
    この全ホールの舞台前方中央付近での測定点の実測値の平均値から±1シグマ区間の間を目標値として設定する。
    

%=====================================================================================

  \section{生成音場の調整方法}
    
    \subsection{方向別STの調整方法}
    実測した方向別STの値をもとに、生成する音場の方向別STを調整する。$\mathrm{ST_{Early,dir}}$は後方が大きく、
    $\mathrm{ST_{Late,dir}}$は全方向から満遍なく反射音が到来していることがわかるため、このような方向特性を
    実現することができるようにマイクとスピーカーの配置と結線を行った。実験室のシステム機器配置を
    図\ref{fig:実験室におけるAFCのシステム機器配置}に、これらのルーティングを表\ref{tab:マイクとスピーカーのルーティング}に示す。


    \begin{table}[H]
      \centering
      \caption{マイクとスピーカーのルーティング}
      \label{tab:マイクとスピーカーのルーティング}

      \begin{tabular}[H]{|c||cccc|cccc|cccc|cccc|cccc|}
        \hline
        system &
        \multicolumn{4}{c|}{
          \begin{tabular}{c}
            1 EMRなし\\
            初期反射音
          \end{tabular}
        }& 
        \multicolumn{4}{c|}{
          \begin{tabular}{c}
            2 EMRあり\\
            後期反射音
          \end{tabular}
        }&
        \multicolumn{4}{c|}{
          \begin{tabular}{c}
            3 EMRあり\\
            後期反射音
          \end{tabular}}&
        \multicolumn{4}{c|}{
          \begin{tabular}{c}
            4 EMRなし\\
            初期反射音
          \end{tabular}
        }& 
        \multicolumn{4}{c|}{
          \begin{tabular}{c}
            5 EMRなし\\
            後期反射音
          \end{tabular}
        } \\ \hline
        mic  & 1 & 2 & 3 & 4 & 5 & 6 & 7 & 8 & 5 & 6 & 7 & 8 & 5 & 6 & 7 & 8 & 5 & 6 & 7 & 8 \\ \hline \hline
        Sp.1 &  &  &  &  &    &  &  & x&   &  &  &  &   &  &  &  &   &  &  & \\ 
        Sp.2 &  &  &  &  &    &  &  &  &   &  &  &  &   &  &  &  &   &  &  & \\
        Sp.3 &  &  &  &  &    & x&  &  &   &  &  &  &   &  &  &  &   &  &  & \\
        Sp.4 &  & o&  &  &    &  &  &  &   &  &  &  &   &  &  &  &   &  &  & \\
        Sp.5 &  &  &  &  &   x&  &  &  &   &  &  &  &   &  &  &  &   &  &  & \\
        Sp.6 &  &  &  &  &    &  &  &  &   &  &  &  &   &  &  &  &   &  &  & \\
        Sp.7 &  &  &  &  &    &  & x&  &   &  &  &  &   &  &  &  &   &  &  & \\
        Sp.8 &  &  & o&  &    &  &  &  &   &  &  &  &   &  &  &  &   &  &  & \\ \hline
        Sp.9 &  &  &  &  &    &  &  & x&   &  &  &  &   &  &  &  &   &  &  & \\
        Sp.10 & &  &  & o&    &  &  &  &   &  &  &  &   &  &  &  &   & o&  & \\
        Sp.11 & &  & o&  &    &  &  &  &   &  &  &  &   &  &  & o&   &  &  & \\
        Sp.12 & &  &  &  &    &  &  &  &   &  &  &  &   & o&  &  &  o&  &  & \\
        Sp.13 & & o&  &  &   x& x&  &  &   &  &  &  &  o&  &  &  &   &  &  & \\
        Sp.14 & o& &  &  &    &  &  &  &   &  &  &  &   &  &  &  &   &  & o& \\
        Sp.15 & &  &  &  &    &  & x&  &   &  &  &  &   &  &  &  &   &  &  & o\\
        Sp.16 & &  &  &  &    &  &  &  &   &  &  &  &   &  & o&  &   &  &  & \\ \hline
        Sp.17 & &  &  &  &    &  &  &  &   x& &  &  &   &  &  &  &   &  &  & \\
        Sp.18 & &  &  & o&    &  &  &  &   & x&  &  &   &  &  &  &   &  &  & \\
        Sp.19 &o&  &  &  &    &  &  &  &   &  & x&  &   &  &  &  &   &  &  & \\
        Sp.20 & &  &  &  &    &  &  &  &   &  &  & x&   &  &  &  &   &  &  & \\ \hline
        
      \end{tabular}
    \end{table}

    \begin{figure}[H]
      \begin{minipage}[b]{1\linewidth}
          \centering
          \includegraphics[width=.6\linewidth]{images/afcEquipArrayVertical.png}
          \subcaption{断面図}
          \label{fig:AFC機器配置断面図}
      \end{minipage}
  
      \begin{minipage}[b]{0.5\linewidth}
          \centering
          \includegraphics[width=.9\linewidth]{images/afcEquipArrayMic.png}
          \subcaption{マイクロホン配置}
          \label{fig:マイクロホン配置}
      \end{minipage}%
      \begin{minipage}[b]{0.5\linewidth}
        \centering
        \includegraphics[width=.9\linewidth]{images/afcEquipArraySp3.png}
        \subcaption{スピーカー配置 三層目}
        \label{fig:スピーカー配置 三層目}
        \vfill
      \end{minipage}
  
      \begin{minipage}[b]{0.5\linewidth}
          \centering
          \includegraphics[width=.9\linewidth]{images/afcEquipArraySp2.png}
          \subcaption{スピーカー配置 二層目}
          \label{fig:スピーカー配置 二層目}
      \end{minipage}%
      \begin{minipage}[b]{0.5\linewidth}
        \centering
        \includegraphics[width=.9\linewidth]{images/afcEquipArraySp1.png}
        \subcaption{スピーカー配置 一層目}
        \label{fig:スピーカー配置 一層目}
        \vfill
      \end{minipage}
      \caption{実験室におけるAFCのシステム機器配置}
      \label{fig:実験室におけるAFCのシステム機器配置}
  \end{figure}

    また、響きの可変幅を十分に確保できるよう、重畳するFIRフィルタには図\ref{fig:FIRフィルタを測定したコンサートホール}に
    示す2000席程度の豊かな響きを持つコンサートホールにて実測したインパルス応答を用いた。初期反射音制御部には舞台付近で
    測定したインパルス応答を、後期反射音制御部には舞台から遠方にて測定したインパルス応答を用いた。それぞれについて、ともに
    異なる4点で測定したインパルス応答を用いており、これは完全に同じインパルス応答を用いることによってAFCシステム内の
    ループごとの周波数特性が近くなることでカラレーションに対する安定性が低下するのを防ぐことを目的としている。
    重畳に用いたインパルス応答の時間波形を図\ref{fig:初期反射音の重畳に用いたFIRフィルタ}および
    図\ref{fig:後期反射音の重畳に用いたフィルタ}に示す。

    \begin{figure}[H]
      \centering
      \includegraphics[width=.6\linewidth]{images/convolutedIrHall.jpg}
      \caption{FIRフィルタを測定したコンサートホール}
      \label{fig:FIRフィルタを測定したコンサートホール}
    \end{figure}

    \begin{figure}[H]
      \begin{minipage}[b]{.5\linewidth}
          \centering
          \includegraphics[width=.9\linewidth]{images/convolutedIr/ER1.png}
          \subcaption*{FIRフィルタ 1}
      \end{minipage}%
      \begin{minipage}[b]{.5\linewidth}
          \centering
          \includegraphics[width=.9\linewidth]{images/convolutedIr/ER2.png}
          \subcaption*{FIRフィルタ 2}
      \end{minipage}

      \begin{minipage}[b]{.5\linewidth}
          \centering
          \includegraphics[width=.9\linewidth]{images/convolutedIr/ER3.png}
          \subcaption*{FIRフィルタ 3}
      \end{minipage}%
      \begin{minipage}[b]{.5\linewidth}
          \centering
          \includegraphics[width=.9\linewidth]{images/convolutedIr/ER4.png}
          \subcaption*{FIRフィルタ 4}
      \end{minipage}

      \centering
      \caption{初期反射音の重畳に用いたFIRフィルタ}
      \label{fig:初期反射音の重畳に用いたFIRフィルタ}
    \end{figure}

    \begin{figure}[H]
      \begin{minipage}[b]{.5\linewidth}
          \centering
          \includegraphics[width=.9\linewidth]{images/convolutedIr/REV1.png}
          \subcaption*{FIRフィルタ 1}
      \end{minipage}%
      \begin{minipage}[b]{.5\linewidth}
          \centering
          \includegraphics[width=.9\linewidth]{images/convolutedIr/REV2.png}
          \subcaption*{FIRフィルタ 2}
      \end{minipage}

      \begin{minipage}[b]{.5\linewidth}
          \centering
          \includegraphics[width=.9\linewidth]{images/convolutedIr/REV3.png}
          \subcaption*{FIRフィルタ 3}
      \end{minipage}%
      \begin{minipage}[b]{.5\linewidth}
          \centering
          \includegraphics[width=.9\linewidth]{images/convolutedIr/REV4.png}
          \subcaption*{FIRフィルタ 4}
      \end{minipage}

      \centering
      \caption{後期反射音の重畳に用いたFIRフィルタ}
      \label{fig:後期反射音の重畳に用いたフィルタ}
    \end{figure}

    $\mathrm{ST_{Early,dir}}$および$\mathrm{ST_{Late,dir}}$を目標値に近づけて制御するためには、初期反射音と
    後期反射音の成分をある程度独立に制御することが必要となる。初期反射音制御部のために付加したエネルギーが後期反射音の
    評価区間に漏れ出す量を少なくするため、初期反射音制御部では重畳するFIRフィルタにフェードアウトを設定することにより
    信号長を$\SI{120}{ms}$に限定した。
    
    また、後期反射音制御部では、反射音の付加に長い遅れ時間を加えることで$\mathrm{ST_{Early,dir}}$の評価区間に入るエネルギーを
    減らすことができるが、遅れ時間が大きすぎるとエコー障害が生じる恐れがある。本研究では、$\SI{45}{ms}$の遅れ時間を
    付け加えることにより、聴感的な自然さを保ちつつ$\mathrm{ST_{Early,dir}}$への影響を低減した。

    \subsection{その他の音響特性の調整方法}
      \subsubsection{残響時間の調整}
      本実験で用いた実験室における音響測定では、部屋の広さによる制約により、直接音の減衰が大きいため残響時間の評価として
      よく用いられる $\mathrm{T_{20}}$ または $\mathrm{T_{30}}$ を適切に測定することができない。
      そこで本研究では、$\mathrm{ST}$の測定条件によるインパルス応答の測定結果からエネルギーの減衰曲線を描き、
      $\SI{-15}{dB}$から$\SI{-45}{dB}$までの減衰曲線の傾きを読むことにより残響時間を求めた。
      生成した音場の残響時間は通常よくあるコンサートホールでの残響時間よりもかなり長くなる傾向があったため、
      後期反射音制御部で重畳するFIRフィルタにフェードアウトを設定することで残響時間を短くし、$\SI{1.8}{秒}$程度
      の自然な長さになるよう調整した。
      
      \subsubsection{周波数特性の調整}
      オクターブバンドごとのSTおよび残響時間の値が極端にばらつくことを防ぐため、測定とAFCシステム内部の
      イコライザの調整を繰り返して周波数特性を調整した。

%=====================================================================================

\section{生成した音場の特性}

  \subsection{基準音場}
  コンサートホールのステージを模して生成した基準音場の方向別STを図\ref{fig:基準音場の方向別ST}に示す。
  各方向での目標値からの差はすべて$\SI{1}{dB}$未満であり、およそ目標値の通りの方向別STを実現することが
  できた。
  
  残響時間およびSTを図\ref{fig:基準音場の残響時間とST}に示す。残響時間および$\mathrm{ST_{Late,dir}}$は
  すべてのオクターブバンドでほぼフラットな周波数特性となった。$\mathrm{ST_{Early,dir}}$は室自体の持つ特性
  が影響していると考えられ、完全な制御はできていないが、平均値からの偏差はすべてのオクターブバンドで$\SI{1}{dB}$
  程度に収めることができた。
  
  減衰曲線を図\ref{fig:基準音場の減衰曲線}に示す。AFCで生成した音場では、エネルギーの付加によって特に響きの
  後期反射音側で減衰曲線が盛り上がる場合があるが、今回生成した音場ではこのような現象は見られず、実際の空間と
  同様の直線的な減衰が確認できた。

  \begin{figure}[H]
    \begin{minipage}
      [b]{.5\linewidth}
      \centering
      \includegraphics[width=1\linewidth]{images/experimentField/withLegend/baseEarlyOnTarget4.png}
      \subcaption*{$\mathrm{ST_{Early,dir}}$}
    \end{minipage}%
    \begin{minipage}
      [b]{.5\linewidth}
      \centering
      \includegraphics[width=1\linewidth]{images/experimentField/withLegend/baseLateOnTarget4.png}
      \subcaption*{$\mathrm{ST_{Late,dir}}$}
      \end{minipage}
    \caption{基準音場の方向別ST}
    \label{fig:基準音場の方向別ST}
  \end{figure}

  \begin{figure}[H]
    \centering
    \includegraphics[width=.8\linewidth]{images/fieldO_stRt.png}
    \caption{基準音場の残響時間とST}
    \label{fig:基準音場の残響時間とST}
  \end{figure}

  \begin{figure}[H]
    \centering
    \includegraphics[width=.8\linewidth]{images/fieldO_EnergyDecayCurve.png}
    \caption{基準音場の減衰曲線}
    \label{fig:基準音場の減衰曲線}
  \end{figure}

  %=====================================================================================

  \subsection{方向の偏りをつけた音場}
  基準音場に対して、初期反射音・後期反射音のそれぞれについて、前方からの反射を強めた音場と後方からの反射を
  強めた音場を生成した。初期反射音の前方を強めた音場を音場A、初期反射音の後方を強めた音場を音場B、後期反射音の
  前方を強めた音場を音場C、後期反射音の後方を強めた音場を音場Dとする。生成した音場の方向別STを
  図\ref{fig:音場Aの方向別ST}から図\ref{fig:音場Dの方向別ST}に示す。

  \begin{figure}[H]
    \begin{minipage}[b]{.5\linewidth}
        \centering
        \includegraphics[width=1\linewidth]{images/experimentField/withLegend/expAEarly.png}
        \subcaption*{$\mathrm{ST_{Early,dir}}$\\前方強め}
    \end{minipage}%
    \begin{minipage}[b]{.5\linewidth}
        \centering
        \includegraphics[width=1\linewidth]{images/experimentField/withLegend/expALate.png}
        \subcaption*{$\mathrm{ST_{Late,dir}}$\\基準音場と同じ}
    \end{minipage}
    \caption{音場Aの方向別ST}
    \label{fig:音場Aの方向別ST}
  \end{figure}

  \begin{figure}[H]
    \begin{minipage}[b]{.5\linewidth}
      \centering
      \includegraphics[width=1\linewidth]{images/experimentField/withLegend/expBEarly.png}
      \subcaption*{$\mathrm{ST_{Early,dir}}$\\後方強め}
    \end{minipage}%
    \begin{minipage}[b]{.5\linewidth}
        \centering
        \includegraphics[width=1\linewidth]{images/experimentField/withLegend/expBLate.png}
        \subcaption*{$\mathrm{ST_{Late,dir}}$\\基準音場と同じ}
    \end{minipage}
    \caption{音場Bの方向別ST}
    \label{fig:音場Bの方向別ST}
  \end{figure}

  \newpage

  \begin{figure}[H]
    \begin{minipage}[b]{.5\linewidth}
      \centering
      \includegraphics[width=1\linewidth]{images/experimentField/withLegend/expCEarly.png}
      \subcaption*{$\mathrm{ST_{Early,dir}}$\\基準音場と同じ}
    \end{minipage}%
    \begin{minipage}[b]{.5\linewidth}
        \centering
        \includegraphics[width=1\linewidth]{images/experimentField/withLegend/expCLate.png}
        \subcaption*{$\mathrm{ST_{Late,dir}}$\\前方強め}
    \end{minipage}
    \caption{音場Cの方向別ST}
    \label{fig:音場Cの方向別ST}
  \end{figure}

  \begin{figure}[H]
    \begin{minipage}[b]{.5\linewidth}
        \centering
        \includegraphics[width=1\linewidth]{images/experimentField/withLegend/expDEarly.png}
        \subcaption*{$\mathrm{ST_{Early,dir}}$\\基準音場と同じ}
    \end{minipage}%
    \begin{minipage}[b]{.5\linewidth}
        \centering
        \includegraphics[width=1\linewidth]{images/experimentField/withLegend/expDLate.png}
        \subcaption*{$\mathrm{ST_{Late,dir}}$\\後方強め}
    \end{minipage}
    \caption{音場Dの方向別ST}
    \label{fig:音場Dの方向別ST}
  \end{figure}

%=====================================================================================




\clearpage

% 参考文献
% 参考文献の箇所にインプットしてください。

% 分割ファイル内でのみ,bibliographyを読み込みます。

\expandafter\ifx\csname ifdraft\endcsname\relax

  % bibliographyを展開する

  \bibliographystyle{junsrt}
  \bibliography{ref.bib}% 同じディレクトリ内のbibファイルのみを参照可能

\fi

\end{document}
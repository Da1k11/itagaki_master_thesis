\documentclass[11pt,a4j]{jreport}

\usepackage{comment}
\usepackage{float}
\usepackage{color}
\usepackage{multicol}
\usepackage{multirow}
\usepackage[dvipdfmx]{pict2e}
\usepackage{wrapfig}
\usepackage{graphicx}
\usepackage{bm}
\usepackage{url}
\usepackage{underscore}
\usepackage{colortbl}
\usepackage{tabularx}
\usepackage{fancyhdr}
\usepackage{ulem}
\usepackage{cite}
\usepackage{amsmath,amssymb,amsfonts}
\usepackage{algorithmic}
\usepackage{textcomp}
\usepackage{xcolor}
\usepackage[ipaex]{pxchfon}
\usepackage{pdfpages}
\usepackage{subcaption}
\usepackage{array}
\usepackage{adjustbox}
\usepackage{lipsum}

\usepackage[number-unit-product=~]{siunitx}

\usepackage[top=30truemm,bottom=30truemm,left=25truemm,right=25truemm]{geometry}

\renewcommand{\arraystretch}{1.2}

\begin{document}

\chapter{おわりに}

%=====================================================================================

\section{まとめ}

コンサートホールのステージ上への反射音の到来方向を評価する物理指標を構築し、その実測結果をもとに、
半無響室に導入した音場生成システムを用いて反射音の到来方向を制御した実験音場を生成した。生成した音場において
混声四部のカルテットによる演奏実験を行った。その結果、後期反射音の前方から供給量の増加が演奏の印象を、
後方からの供給量の増加が響きと空間の印象を向上させる可能性が示唆された。

%=====================================================================================

\section{今後の展望}
実際のコンサートホールにおける後期反射音はほとんど拡散しており、建築的な手法で到来方向の偏りを大きくすることは
容易ではないと考えられる。しかし、本研究で用いたAFCのような自然な響きを付加することのできる電気音響設備を
用いて方向特性に偏りのある後期反射音を演奏者に対して供給することは可能である。ステージ付近には演奏音を収集する
ためのマイクロホンが置かれており、ステージ上への反射音供給がハウリングやカラレーションの原因になる場合を避けるため
、現在ではAFCを用いた反射音成分の付加の多くは客席に対して行われているが、ステージ上にも電気音響設備による
反射音成分を演奏者に対して積極的に供給することで、演奏者にとってより好ましい音環境を作ることができるようになる可能性がある。

また、本研究において、AFCの調整の仕方を工夫することにより所望の音場をある程度自由に生成することができた。
音響指標値を制御した御場を自由に生成できれば、演奏実験はもちろん、AFCを用いて様々な建築音環境に関する
主観評価実験を行うことができる可能性がある。
そのような用途でAFCを用いることを考える場合、現時点ではAFCの調整はその大部分を手動のパラメータ調整に頼っており、
物理音響指標値等を所望の値に調整して音場を生成することは容易ではないことが課題としてあげられるが、
モデルベースド制御とモデル上でのAFCのパラメータの探索および最適化等を行うことで、現在手動で調整しているAFCの
パラメータを自動で調整することができるようになる可能性がある。

%=====================================================================================

\clearpage

% 参考文献
% 参考文献の箇所にインプットしてください。

% 分割ファイル内でのみ,bibliographyを読み込みます。

\expandafter\ifx\csname ifdraft\endcsname\relax

  % bibliographyを展開する

  \bibliographystyle{junsrt}
  \bibliography{ref.bib}% 同じディレクトリ内のbibファイルのみを参照可能

\fi

\end{document}
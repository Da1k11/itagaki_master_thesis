\documentclass[11pt,a4j]{jreport}

\usepackage{comment}
\usepackage{float}
\usepackage{color}
\usepackage{multicol}
\usepackage{multirow}
\usepackage[dvipdfmx]{pict2e}
\usepackage{wrapfig}
\usepackage{graphicx}
\usepackage{bm}
\usepackage{url}
\usepackage{underscore}
\usepackage{colortbl}
\usepackage{tabularx}
\usepackage{fancyhdr}
\usepackage{ulem}
\usepackage{cite}
\usepackage{amsmath,amssymb,amsfonts}
\usepackage{algorithmic}
\usepackage{textcomp}
\usepackage{xcolor}
\usepackage[ipaex]{pxchfon}
\usepackage{pdfpages}
\usepackage{subcaption}
\usepackage{array}
\usepackage{adjustbox}
\usepackage{lipsum}

\usepackage[number-unit-product=~]{siunitx}

\usepackage[top=30truemm,bottom=30truemm,left=25truemm,right=25truemm]{geometry}

\renewcommand{\arraystretch}{1.2}

\begin{document}

\chapter{おわりに}

%=======================================================================
\section{まとめ}
本研究では、コンサートホールのステージ上への反射音の到来方向を評価する物理指標を構築し、その実測結果をもとに音場の生成を行い、反射音の到来方向特性が合唱者に与える影響について検討を行うことを目的とした。

第1章では、研究の背景としてコンサートホールとその研究について概観し、コンサートホール音響学の一領域であるステージ音響学と、特に反射音の方向特性に関する既往研究について整理した。さらに、本研究の目的として反射音到来方向が演奏者に与える影響について検討することを述べ、また本研究で用いる実験システムであるAFCについて簡単に紹介した。

第2章では、反射音の到来方向について物理的に評価する方法として方向別STを提案し、コンサートホールにおいてその実測を行なった。その結果から、$\mathrm{ST_{Early,dir}}$は、全ホールに共通して方向別の偏差が見られ、特にすべての前方の値が相対的に低く、壁面側で値が大きくなっているのに対し、$\mathrm{ST_{Late,dir}}$は方向的に拡散しており、方向別の偏差が小さいことを示した。また、ISOにおける測定条件の範囲であっても、壁面の付近では$10$〜$\SI{20}{\ms}$に到来するエネルギーが反射音のエネルギーのうちで支配的であることを示し、方向別STの評価では標準的な演奏位置でもある舞台中央・前方付近での測定値を用いて評価することとした。さらに、これらの測定結果を踏まえて、演奏実験を行うための生成音場の目標値を設定した。

第3章では、本研究で用いた音場生成システムであるAFCの仕組み、およびそれを導入した実験室のシステム機器構成について説明し、AFCを用いた音場の生成方法と生成された音場の音響特性について述べた。また、AFCにおける調整項目や本研究において用いた設定値についても示した。

第4章では、第3章で得られた音場を用いて、混声四部のカルテットによる演奏実験を行い、その方法、結果および考察について示した。その結果、被験者ごとの評価の個人差が大きい中でも、全体的な傾向として、後期反射音の前方から供給量の増加が演奏の印象を、後方からの供給量の増加が響きと空間の印象を向上させる可能性を示唆した。

%=======================================================================

\section{今後の展望}
実際のコンサートホールにおける後期反射音はほとんど拡散しており、建築的な手法で到来方向の偏りを大きくすることは容易ではないと考えられる。しかし、本研究で用いたAFCのような自然な響きを付加することのできる電気音響設備を用いて方向特性に偏りのある後期反射音を演奏者に対して供給することは可能である。ステージ付近には演奏音を収集するためのマイクロホンが置かれており、ステージ上への反射音供給がハウリングやカラレーションの原因になる場合を避けるため、現在ではAFCを用いた反射音成分の付加の多くは客席に対して行われているが、ステージ上にも電気音響設備による反射音成分を演奏者に対して積極的に供給することで、演奏者にとってより好ましい音環境を作ることができるようになる可能性がある。

また、本研究において、AFCの調整の仕方を工夫することにより所望の音場をある程度自由に生成することができた。音響指標値を制御した御場を自由に生成できれば、演奏実験はもちろん、AFCを用いて様々な建築音環境に関する主観評価実験を行うことができる可能性がある。
そのような用途でAFCを用いることを考える場合、現時点ではAFCの調整はその大部分を手動のパラメータ調整に頼っており、物理音響指標値等を所望の値に調整して音場を生成することは容易ではないことが課題としてあげられるが、モデルベースド制御とモデル上でのAFCのパラメータの探索および最適化等を行うことで、現在手動で調整しているAFCのパラメータを自動で調整することができるようになる可能性がある。

%=======================================================================

\clearpage

% 参考文献
% 参考文献の箇所にインプットしてください。

% 分割ファイル内でのみ,bibliographyを読み込みます。

\expandafter\ifx\csname ifdraft\endcsname\relax

  % bibliographyを展開する

  \bibliographystyle{junsrt}
  \bibliography{ref.bib}% 同じディレクトリ内のbibファイルのみを参照可能

\fi

\end{document}